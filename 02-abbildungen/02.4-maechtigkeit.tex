\subsection{Mächtigkeit von Mengen von Abbildung}
Seien $A$ und $B$ Mengen.
Dann bezeichnet $B^A$ oder $\Map(A,B)$ die Menge aller Abbildungen von $A$
nach $B$. \paragraph*{Satz:}
Für $A, B$ endliche Mengen gilt:
$$ |B^A| = {|A|}^{|B|} $$

\paragraph*{bijektive Abbildungen}
Sei $|A|=|B|$ endlich. Die Mächtigkeit der Menge aller bijektiven
Abbildungen von A nach B entspricht der Anzahl der Permutationen auf
einer |A|-elementigen Menge.
\begin{align*}
  A &\rightarrow B \\
  (|A|)! &= (|B|)!
\end{align*}

\paragraph*{injektive Abbildungen}
Seien |A| und |B| endlich. Die Mächtigkeit der Menge aller injektiven
Abbildungen von A nach B entspricht einer \emph{Variation} ohne Zurücklegen:
\begin{align*}
  A &\rightarrow B \\
  a &= |A| \; b = |B| \\
  {{b}\choose{a}} * a!
  &= \frac{b!}{(b-a)!}
\end{align*}

\subparagraph*{Anmerkung:} Für den Sonderfall $|A|=|B|$ führt die Formel auf die Formel für die bijektiven Abbildungen zurück. Das bedeutet, dass im Fall $|A|=|B|$ endlich jede Injektion
automatisch auch eine Bijektion ist.

\paragraph*{surjektive Abbildung}
Seien |A| und |B| endlich. Die Mächtigkeit der Menge aller surjektiven
Abbildungen von A nach B lässt sich mithilfe der
Stirlingzahl 2. Art ($S_{n,r}$) berechnen.
\begin{align*}
  A &\rightarrow B \\
  a &= |A| \; b = |B| \\
  b!    &\cdot S_{(a,b)}\\
  S(a,b)&=\frac{1}{b!}\sum_{j=1}^{b}(-1)^{b-j}{b \choose j}j^a
\end{align*}