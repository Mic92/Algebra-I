\subsection{Mächtigkeit von Mengen von Abbildungen}
Seien $A$ und $B$ Mengen.
Dann bezeichnet $B^A$ oder $\Map(A,B)$ die Menge aller Abbildungen von $A$
nach $B$. \paragraph*{Satz:}
Für $A, B$ endliche Mengen gilt:
$$ |B^A| = {|B|}^{|A|} $$

\subsubsection{Formeln zur Berechnung der Mächtigkeit von Mengen von Abbildungen}
\paragraph*{bijektive Abbildungen}
Sei $|A|=|B|$ endlich. Die Mächtigkeit der Menge aller bijektiven
Abbildungen von A nach B entspricht der Anzahl der Permutationen auf
einer |A|-elementigen Menge.
\begin{align*}
  A &\rightarrow B \\
  |Bij(A,B)| = (|A|)! &= (|B|)!
\end{align*}

\paragraph*{injektive Abbildungen}
Seien |A| und |B| endlich. Die Mächtigkeit der Menge aller injektiven
Abbildungen von A nach B entspricht einer \emph{Variation} ohne Zurücklegen:
\begin{align*}
  A &\rightarrow B \\
  a &= |A| \; b = |B| \\
  |Inj(A,B)| =  
{{b}\choose{a}} * a!
  &= \frac{b!}{(b-a)!}
\end{align*}

\subparagraph*{Anmerkung:} Für den Sonderfall $|A|=|B|$ führt die Formel auf die Formel für die bijektiven Abbildungen zurück. Das bedeutet, dass im Fall $|A|=|B|$ endlich jede Injektion
automatisch auch eine Bijektion ist:
$$|A|=|B| \Longleftrightarrow Inj(A,B) = Bij(A,B) = Surj(A,B)$$
z.B.: $ Inj(A,A) = Bij(A,A) =: Perm(A) $
(die Permutation einer Menge ist die Bijektion einer Menge in sich selbst)

\paragraph*{surjektive Abbildung}
Seien |A| und |B| endlich. Die Mächtigkeit der Menge aller surjektiven
Abbildungen von A nach B lässt sich mithilfe der
Stirlingzahl 2. Art ($S_{n,r}$) berechnen.
\begin{align*}
  A &\rightarrow B \\
  a &= |A| \; b = |B| \\
  |Surj(A,B)| = 
  b!    &\cdot S_{(a,b)}\\
  S(a,b)&=\frac{1}{b!}\sum_{j=1}^{b}(-1)^{b-j}{b \choose j}j^a
\end{align*}

\subparagraph*{Bemerkung:} Die Formel für surjektive Abbildung sollte als Ergänzung zum Vorlesungsstoff verstanden werden. 
Sie wird(soweit wir bisher wissen) nicht als bekannt vorrausgesetzt.

\subparagraph{Weitere Bemerkungen}
Es gilt:
$ |A| = n \Longleftrightarrow Bij( \{1, ... , n\},A) \neq \emptyset$

\subsubsection{Herleitung einer rekursiven Formel für |Surj(A,B)| }
Idee: $$ B^A =  \underset {{X}\subseteq{B}} {\mathbin{\dot{\cup}}} {\{ f\in B^A | f(A) = X \}} $$

Offensichtlich\footnote{Das ist hier wohl eher ironisch gemeint} gilt: $$ | {f \in B^A | f(A) = X} | = |Surj(A,X)| = s(|A|, |X|) $$