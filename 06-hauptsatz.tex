Jede natürliche Zahl $n \ge 2$ besitzt eine eineindeutige
Primfaktorzerlegung.

Die dazugehörige Abbildung $\kappa:= N^P \rightarrow N_+, \alpha
\mapsto \Pi_{p \in P} p^{\alpha(p)}$, welche die Primzahlen auf die
natürliche Zahl abbildet, ist bijektiv.

Der kleinste gemeinsame Teiler und das größte gemeinsame Vielfache
lassen folgender maßen berechnen:

a und b sind zwei Zahlen, welche sich in jeweils 2 Multimenge von Primzahlen
zerlegen lassen.
\begin{align*}
  a &= p_1^{a1} * p_2^{a2} * p_3^{a3}\\
  b &= p_1^{b1} * p_2^{b2} * p_3^{b3}\\
\end{align*}
Der größten gemeinsamen Teiler setzt sich aus dem Produkt der
Primzahlen zusammen, welche in beiden Mengen enthalten sind
(Durchschnitt). Beim kleinsten gemeinsamen Vielfachen wird die
Vereinigungsmenge beider Primzahlmengen multipliziert.
\begin{align*}
  ggT(a,b) &= p_1^{min(a1,b1)} * p_1^{min(a2,b2)} * p_1^{min(a3,b3)}\\
  kgV(a,b) &= p_1^{max(a1,b1)} * p_1^{max(a2,b2)} * p_1^{max(a3,b3)}\\
\end{align*}

Beispiel:
\begin{align*}
  600 &= 2^3 * 3^1 * 5^2\\
  160 &= 2^3 * 4^1 * 5^1\\
  ggT(600, 160) &= 2^3 * 5^1\\
  kgV(600, 160) &= 2^3 * 3^1 * 4^1 * 5^2\\
\end{align*}

%%% Local Variables:
%%% mode: latex
%%% TeX-master: "script"
%%% End:
