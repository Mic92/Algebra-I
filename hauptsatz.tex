Jede natürliche Zahl $n \ge 2$ besitzt eine eineindeutige
Primfaktorzerlegung.
\begin{align*}
\mathbb N^{(A)} := \{ \alpha \in \mathbb N^{(A)} | \supp(\alpha) \text{endlich}
\}
\end{align*}

\paragraph{Allgemein:} $B^A$ ist die Menge aller Abbildungen von $A$ nach $B$.
Die Abbildung $f: A \rightarrow B, x \mapsto f x$ ordnet jedem $x \in A$ genau
ein $f x \in B$ zu.

$A = P = $ sei die Menge der Primzahlen.
Die dazugehörige Abbildung $\kappa:= \mathbb N^P \rightarrow \mathbb N_+,
\alpha \mapsto \prod \limits_{p \in P} p^{\alpha(p)}$, welche die Primzahlen
auf die natürlichen Zahlen abbildet, ist bijektiv.

\paragraph{Satz}: Für $f: B \rightarrow \mathbb N$ ist $\prod \limits_{p \in B}
f(p)$ das Produkt über alle $f(p)$, wo $p$ alle Elemente aus $B$ durchläuft.
Genauer gesagt ist diese Funktion \glqq rekursiv definiert\grqq: Sei $b \in B$.
Dann sei $\prod \limits{p \in B} f p := f b \cdot \prod \limits_{p \in B}
f p_{ \{b \}}$, wo $b$ unendliche Menge sei (hängt nicht von $b \in B$ ab), da
$(\mathbb N_+, \cdot, 1)$ ein kommutatives Monoid ist. Analog verhält es sich
mit der Summe $\sum \limits_{p \in B} fp$.

\subsection{Kleinster gemeinsamer Teiler und größtes gemeinsames Vielfaches}

Der kleinste gemeinsame Teiler und das größte gemeinsame Vielfache
lassen sich wie folgt berechnen:

$a$ und $b$ sind zwei Zahlen, welche sich in jeweils 2 Multimengen von
Primzahlen zerlegen lassen.
\begin{align*}
  a &= p_1^{a1} * p_2^{a2} * p_3^{a3}\\
  b &= p_1^{b1} * p_2^{b2} * p_3^{b3}\\
\end{align*}
Der größte gemeinsame Teiler setzt sich aus dem Produkt der
Primzahlen zusammen, welche in beiden Mengen enthalten sind
(Durchschnitt). Beim kleinsten gemeinsamen Vielfachen wird die
Vereinigungsmenge beider Primzahlmengen multipliziert.
\begin{align*}
  \ggT(a,b) &= p_1^{\min(a1,b1)} * p_1^{\min(a2,b2)} * p_1^{\min(a3,b3)}\\
  \kgV(a,b) &= p_1^{\max(a1,b1)} * p_1^{\max(a2,b2)} * p_1^{\max(a3,b3)}\\
\end{align*}

Beispiel:
\begin{align*}
  600 &= 2^3 * 3^1 * 5^2\\
  160 &= 2^3 * 4^1 * 5^1\\
  \ggT(600, 160) &= 2^3 * 5^1\\
  \kgV(600, 160) &= 2^3 * 3^1 * 4^1 * 5^2\\
\end{align*}

\paragraph{Satz:} Für alle $m, n \in \mathbb N_+$ gilt:
\begin{align*}
m \cdot n &= (m \lor_{\tau} n) \cdot (m \land_{\tau} n)\\
&= \kgV(m,n) \cdot \ggT(m,n)
\end{align*}

\paragraph{Beweis} mit der \glqq Gemüseformel\grqq:
\begin{align*}
\alpha + \beta = \alpha \lor \beta + \alpha \land \beta
\end{align*}
Seien $\alpha, \beta \in \mathbb N^{(P)}$ mit $\kappa \alpha = m$ \& $\kappa
\beta = n$. Dann gilt $\kappa(\alpha + \beta) = \kappa(\alpha) + \kappa(\beta)$.
Also ist
\begin{align*}
m \cdot n &= \kappa(\alpha + \beta)\\
 &= \kappa(\alpha \lor \beta + \alpha \land
\beta)\\
&= \kappa(\alpha \lor \beta) \cdot \kappa(\alpha \land \beta)\\
&= (\kappa \alpha \lor_{\tau} \kappa \beta) \cdot (\kappa \alpha \land_{\tau}
\kappa \beta)\\
&= (m \lor_{\tau} n) \cdot (m \land_{\tau} n) \qed
\end{align*}

%%% Local Variables:
%%% mode: latex
%%% TeX-master: "script"
%%% End:
