Der Unterschied von Multimengen zu gewöhnlichen Mengen besteht darin,
dass Elemente mehrfach vorkommen können. Sie ist definiert als Element
von $N^A$, wobei $A$ die Grundmenge ist und $N^A$ die Menge aller
Multimengen der Grundmenge $A$ ist. Desweiteren gibt es die geordnete
Menge aller Multimengen $(N^A,\le)$, bei der für alle $\alpha, \beta
\in N^A$ $\alpha \le \beta$ bzw. $\alpha x \le \beta x \text{ für alle
} x \in A$ gilt.

\paragraph{Mengenoperationen:}
$A$ und $B$ sind 2 Multimengen der gleichen Grundmenge.
\begin{enumerate}
\item große Vereinigung/Summe: $(A \uplus B)(a) := A(a) + B(a)$, man
  addiert die Anzahl jedes Elements beider Mengen
\item kleine Vereinigung: $(A \cup B)(a) := max(A(a), B(a))$, man
  nimmt die jeweils größere Anzahl jedes Elements beider Mengen.
\item Durchschnitt: $(A \cap B)(a) := min(A(a), B(a))$, man nimmt die
  jeweils kleinere Anzahl jedes Elements beider Mengen.
\end{enumerate}

--- Beispiel einfügen ---

%%% Local Variables:
%%% mode: latex
%%% TeX-master: "script"
%%% End:
