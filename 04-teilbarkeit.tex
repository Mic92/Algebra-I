% --Teilbarkeit--

Seien $a, b \in \mathbb{N}$. Dann ist $a$ \emph{Teiler von} $b$ genau dann wenn,
es ein $m \in \mathbb{N}$ gibt, so dass gilt: $a \cdot m = b$.

\paragraph{Bezeichnung:} $a \mid b$, $a \le_\tau b$

\paragraph{Beispiel:}
\begin{align*}
3 \mid 6,                    & \text{ denn }     3 \cdot 2 = 6    \\
n \mid 0,                    & \text{ für alle } n \in \mathbb{N} \\
n \le_\tau 0,                 & \text{ für alle } n \in \mathbb{N} \\
0 \mid 0, \text{ } 0 \nmid k & \text{ für alle } k \in \mathbb{N} \setminus \{0\} \\
                             & \text{ (es gibt }  m \in \mathbb{N}: 0 \cdot m = 0 \text{)} \\
1 \mid n,                    & \text{ für alle } n \in \mathbb{N} \\
\end{align*}

\paragraph{Bemerkung:}
$\le_\tau$ ist eine Relation über $\mathbb{N}$:

$\le_\tau \subseteq \mathbb{N} \times \mathbb{N}$.

$(a,b) \in \le_\tau$ gilt genau dann, wenn $a \le_\tau b$.

\paragraph{Eigenschaften von Relationen:}

Sei $R \subseteq A \times A$ eine Relation. Dann heißt $R$:
\begin{itemize}
\item \emph{symmetrisch} fall
  s für $(a, b) \in \mathbb{R}$ stets auch gilt: $(b, a) \in \mathbb{R}$
 \item \emph{antisymmetrisch} falls für $(a, b) \in \mathbb{R}$ und $(b, a) \in \mathbb{R}$  stets folt: $a = b$
 \item \emph{reflexiv} falls $(a, a) \in \mathbb{R}$ gilt für alle: $a \in A$
 \item \emph{transitiv} falls aus $(a, b) \in \mathbb{R}$ und $(b, c) \in \mathbb{R}$ folgt: $(a, c) \in \mathbb{R}$
\end{itemize}

\noindent Eine Ordnung(-srelation) ist eine Relation, die antisymmetrisch,
transitiv und reflexiv ist. \\ \indent {\bf Beispiel:} $\le_\tau, \le$ \\

\noindent Eine Äquivalenz(-relation) ist eine Relation, die symmetrisch,
transitiv und reflexiv ist. \\ \indent {\bf Beispiel:} $=, \equiv$ (mod $n$) \\

\noindent $T_n$ sei die Menge der Teiler von $n (n \in \mathbb{N})$. \\ \indent {\bf Beispiel:} $T_{12} = \{ 1, 2, 3, 4, 6, 12 \}$ \\

\noindent Wir schränken $\le_\tau$ ein auf $T_n$: $\le_\tau \cap (T_n \times T_n) = \le_\tau^n$

\paragraph{Konvention:} wir schreiben $\le_\tau$ statt $\le_\tau$ statt $\le_\tau \cap (A \times A)$, wenn klar ist, worauf sich $\le_\tau$ bezieht, also was $A$ ist. \\

\noindent $\mathbb{T}_n := (T_n, \le_\tau)$ heißt {\bf Teilerverband} von $n$.

\paragraph{Achtung:} $\mathbb{T}_n$ ist ein binäres Relat. Wir haben bereits definiert, was $G (\mathbb{T}_n) = G (T_n, \le_\tau)$ ist:
\indent $G (\mathbb{T}_n) = (T_n, \le_\tau, \sigma, \tau)$ mit $\sigma (a, b) = a, \tau (a, b) = b, \forall (a, b) \in \le_\tau$ \\

\paragraph{Beispiel:} $\mathbb{T}_6 = (T_6, \le_\tau), T_6 = { 1, 2, 3, 6 }$

$G (\mathbb{T}_6) = $ --- Graph einfügen --- \\

\noindent Ist $R$ eine Ordnung über $A$, dann ist $G (A, R)$ eine ungünstige Darstellung. \\ {\bf Besser:} Hasse-Diagramm:

$\le_\tau$ ist eine Relation auf $\mathbb{N}$, die gegeben ist durch: \\
$ \le_\tau = \{ (a, b) \in \le_\tau | \text{ für alle } c \in \mathbb{N} \text{ mit } a \le_\tau c \le_\tau b, (a, c), (c, b) \in \le_\tau \text{ gilt entweder } c = a \text{ oder } c = b \}$

\paragraph{Beispiel:}
\begin{itemize}
 \item $(1, 6) \in \le_\tau$ aber $1 \le_\tau 2 \le_\tau 6, 1 \le_\tau 3 \le_\tau 6$ also ist $(1, 6) \not\in\le_\tau$.
 \item $(2, 6) \in \le_\tau$ und für alle $c \in T_6$ mit $2 \le_\tau c \le_\tau 6, (2, 2), (2, 6), (6, 6)$ gilt $c = 2$ oder $c = 6 \rightarrow (2, 6)$
\end{itemize}

\paragraph{Hassediagramm zu $\mathbb{T}_n$:}

Zeichne $G (T_n, \le_\tau)$, aber so, dass der Knoten b über dem Knoten $a$ liegt, falls $(a, b) \in \le_\tau$ und zeichne Kanten $(a, b)$, also $a$ ----- $b$ statt $a \longrightarrow b$. \\

--- Graph einfügen ---
%%% Local Variables:
%%% mode: latex
%%% TeX-master: "script"
%%% End:
