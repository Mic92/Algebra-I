\subsection{Grundlegendes}
\subsubsection*{Was ist eine Menge?}
Eine Menge ist eine Zusammenfassung unterscheidbarer Objekte zu einer
Gesamtheit. Die Reihenfolge der Elemente ist irrelevant. Jedes Element ist
einzigartig.

Seien A und B Elemente, dann gilt:
\begin{align}
  A = B    &\Leftrightarrow \{A, B\} = \{A\} \\
  A \neq B &\Leftrightarrow \{A, B\} \neq \{A\}
\end{align}
D.h. gleiche Elemente werden in Mengen nur einmal gezählt.
2 Mengen sind genau dann gleich, wenn sie die selben Elemente
enthalten.
\subsubsection*{Besondere Mengen}
Die Menge, die keine Elemente enthält, wird als die \emph{leere Menge}
bezeichnet, das Symbol hierfür ist: $\{\}$ oder ${}\emptyset$.

Die \emph{Potenzmenge} ist die Vereinigung aller Teilmengen einer Menge.
Sie wird mit \(P(A)\) oder \(2^A\) bezeichnet. Jede Potenzmenge
enthält die leere Menge als Element.

\paragraph{Def.:} $P(A):= \{ U| {U}\subseteq{A} \}$
\paragraph{Beispiel:}
\begin{math}
{A = \{1,2,3\} }
\Rightarrow{2^A = \{ \emptyset, \{1\},\{2\},\{3\},\{1,2\},\{1,3\},A \} }
\end{math}