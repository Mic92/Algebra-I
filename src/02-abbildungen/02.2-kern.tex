\subsection{Kern einer Funktion}
Sei $f:{A}\longrightarrow{B}$ eine Abbildungsvorschrift.
Dann ist:
\begin{align*}
   ker(f) := \{(a,b) | f(a)=f(b)\}
\end{align*}
eine Menge, der sogenannte Kern von f.
\subsubsection{Beispiel}
% CS: kriegt das jemand mal vernuenftig formatiert? Bin mit eqnarray oder so nicht weitergekommen :(
$$ f: \mathbb{R} \rightarrow \mathbb{R} : {x}\longmapsto{x^2} $$
$$ ker(f)= \{(a,b) | f(a)=f(b) \} = $$
$$ = \{ (a,b) | a^2 = b^2 \} = $$
$$ = \{ (a,b) | |a| =|b| \} $$