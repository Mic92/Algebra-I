\subsection{Definition}
Seien A und B zwei Mengen.
Dann ist eine \emph{Abbildung} ein eindeutige Vorschrift, die jedem Element aus A genau ein Element aus B zuordnet.
Ein andere Bezeichnung für Abbildung ist \emph{Funktion}.

Die Menge $A$ wird als \emph{Defintionsbereich} bezeichnet.

Die Menge $B$ wird als \emph{Wertebereich} bezeichnet.

Der \emph{Bildbereich} wird definiert als $ f^{ [A] }:=\{b | b=f(a) , a \in A \}$

Von bijektiven Funktionen kann eine \emph{Umkehrfunktion} $f^{-1} : {B}\rightarrow{A} $ gebildet werden.

Diese ist nicht zu verwechseln mit dem \emph{Urbild} die ähnlich geschrieben wird:
$$ f^{-1}(c) = \{a,b,c\} $$ \footnote{Grafik mit den Mengen zum Beispiel wird noch erstellt. Ohne Grafik ist das Beispiel nicht verständlich.}

\subsubsection*{Beispiele}
Hier müssen noch Beispiele eingefügt werden.