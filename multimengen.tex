Der Unterschied von Multimengen zu gewöhnlichen Mengen besteht darin,
dass Elemente mehrfach vorkommen können. Sie ist definiert als Element
von $N^A$, wobei $A$ die Grundmenge ist und $N^A$ die Menge aller
Multimengen der Grundmenge $A$ ist.

\paragraph{Vergleichbarkeit:} Eine Multimengen ist größer gleich einer anderen Multimenge, wenn die
Anzahl aller Elemente größer gleich der anderen Multimenge ist:
$\alpha, \beta \in N^A$, $\alpha \le \beta$ bzw. $\alpha x \le \beta x \text{ für alle
} x \in A$

\paragraph{Notation:} Kleine Multimengen werden mit doppelten
Klammern dargestellt: $\{\{a,b,b,c,c,c\}\}$

Bei großen Multimengen drückt man die Elemente durch 2er-Tupel aus, bei
den der erste Teil das Element und der zweiten die Anzahl des Element
ist: $\{(a,1),(b,2),(c,3)\}$


\paragraph{Mengenoperationen:}
$A$ und $B$ sind 2 Multimengen der gleichen Grundmenge.
\begin{enumerate}
\item große Vereinigung/Summe: $(A + B)(a) := A(a) + B(a)$, man
  addiert die Anzahl jedes Elements beider Mengen
\item kleine Vereinigung: $(A \vee B)(a) := \max(A(a), B(a))$, man
  nimmt die jeweils größere Anzahl jedes Elements beider Mengen.
\item Durchschnitt: $(A \wedge B)(a) := \min(A(a), B(a))$, man nimmt die
  jeweils kleinere Anzahl jedes Elements beider Mengen.
\end{enumerate}

--- Beispiel einfügen ---

%%% Local Variables:
%%% mode: latex
%%% TeX-master: "script"
%%% End:
