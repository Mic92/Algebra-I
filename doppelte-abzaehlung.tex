% Doppelte Abzaehlung
Die doppelte Abzählung ist ein wichtiges Prinzip der Kombinatorik.
Sie besagt das für jede zweistellige Relation (binäre Relation) $R \subseteq A \times B$
gilt:

\begin{align}
  \sum_{a \in A} r_{1}(A) = \sum_{b \in B} r_{2}(B)
\end{align}

Beide Seiten der Gleichung ergeben die Anzahl der Elemente von Relation R: $|R|$.

\paragraph{Beispiel}
Sei $A = {1, 2, 3}$ und $B = {a, b, c, d}$. Die Relation $R \subseteq
A \times B$ ist durch folgende Kreuztabelle gegeben. Wobei ein
Kreuz bedeutet, dass die Spalte und Zeile in Relation stehen.
\begin{align*}
  R =
  \bordermatrix{
    & a & b & c & d \cr
   1& x &   & x &   \cr
   2&   & x & x & x \cr
   3& x & x & x &   \cr
  }
\end{align*}
Es gilt:
\begin{align*}
  r_{1}(1)=2,\; r_{1}(2)=r_{1}(3)=3\\
  r_{2}(a)=r_{2}(b)=2,\; r_{2}(c)=3,\; r_{2}(d)=1\\
\end{align*}
Daraus folgt wie zu erwarten:
\begin{align*}
  r_{1}(1) + r_{1}(2) + r_{1}(3) &= 2 + 3 + 3 = 8\\
  r_{2}(a) + r_{2}(b) + r_{2}(c) + r_{2}(d) &= 2 + 2 + 3 + 1 = 8
\end{align*}


\subsection{Inzidenzstruktur}
Die Inzidenzstruktur $(P,B,I)$ ist ein Tupel bestehend aus:
\begin{itemize}
\item einer Menge P (bezeichnet als Punktmenge)
\item einer Menge B (bezeichnet als Block/Geradenmenge)
\item und einer Inzidenzrelation I (also $I \subset P\times B$ bzw.
  $I = \{(p,b) \in B \times P\}$)
\end{itemize}

\paragraph{taktische Konfiguration}
Eine Inzidenzstruktur $(P,B,I)$ heißt taktische Konfiguration, falls
für alle $b\in B$ und $p \in P$ folgendes gilt:
\begin{itemize}
\item es gibt gleich viele Blöcke/Geraden b pro Punkt p (Verbindungszahl)
\item es gibt gleich viele Punkte p pro Block/Gerade b (Schnittzahl)
\end{itemize}

Beispiel Würfel:

Ein Würfel hat 8 Ecken und 6 Flächen. Pro Ecke gibt es 3 angrenzende
Fläche und pro Fläche 4 Ecken. $3 * 8 = 24 = 6 * 4$

%%% Local Variables:
%%% mode: latex
%%% TeX-master: "script"
%%% End:
