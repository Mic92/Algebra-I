
\subsection{Monoid}
Ein Monoid ist ein Tripel $(M,*,e)$, bestehend aus:
\begin{itemize}
\item einer Menge $M$
\item einer Relation *: $MxM \rightarrow M (a,b) \mapsto a*b$
\item und einem neutralen Element $e \in M$ für das gilt: $e*a = a*e = a$
\end{itemize}

Beispiele:
\begin{itemize}
\item $(N_{0}, +, 0)$
\item $(N, *, 1)$
\end{itemize}

Als Untermonoid $(U,*,e)$ bezeichnet man ein Monoid bei dem die Menge
U Teilmenge von M ist

Ein Monoid ist abelsch, wenn es kommutativ ist. d.h $a*b=b*a$

$(\mathbb{N},\cdot,1)$ ist abelsch da $a \cdot b = b \cdot a$ für
natürliche Zahlen gilt. $(\mathbb{N}^{2x2},\cdot,E_2)$ ($E_2$ ist die
Einheitsmatrix) ist nicht abelsch, da die Matrizenmultiplikation nicht
kommutativ ist.

Satz:  Sei $\mathbb{A} = (A,*,1)$ endliches Monoid, dann sind
äquivalent:
\begin{itemize}
\item $\mathbb{A}$ ist links-kürzbar, d.h $x \cdot y = x \cdot z \Rightarrow
  y = z$
\item $\mathbb{A}$ ist rechts-kürzbar, d.h $x \cdot z = y \cdot z \Rightarrow
  x = y$
\item $\mathbb{A}$ bildet Gruppen, d.h $\forall x \in \mathbb{A}
  \exists y \in A$: $x * y = 1 = y * x$
\end{itemize}

\subsection{Semiring}
Ein Semiring $(M, +, *, 0, 1)$ besteht aus
\begin{itemize}
\item einer nicht leeren Menge M
\item einer Relation (Addition) +: $MxM \rightarrow M$
\item einer Relation (Multiplikation) +: $MxM \rightarrow M$
\item einem Nullelement 0: $0+a = a+0 = a$
\item einem Einselement 1: $1*a = a*1 = a$
\end{itemize}

Beispiele:
\begin{itemize}
\item (N, +, *, 0, 1)
\item Boolescher Halbring $(B, +, *, 0, 1)$ wobei $B=\{0, 1\}$
\item $(Z_{N}, +, *, 0, 1)$ (für RSA relevant)
\end{itemize}


%%% Local Variables:
%%% mode: latex
%%% TeX-master: "script"
%%% End:
