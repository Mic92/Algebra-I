\subsection{Moprhismen zum Vergleich von Graphen}
\subsubsection{Definition}
Seien G und G' zwei Netzwerke(Graphen) sowie $G \overset{\Phi}{\rightarrow} G'$ 
\\Dann ist $\Phi  = (\Phi_{kante}, \Phi_{ecke})$  ein \emph{Morphismus} ein Paar von Abbildungen:
$$ \Phi_{vert}: V \rightarrow V'$$
$$\Phi_{edge}: E \rightarrow E'$$
mit $\tau' \Phi_{edge} e = \Phi_{vert} \tau e$
\\und $ {\sigma '}  \Phi_{edge} e = \Phi_{vert} \sigma e$
% Anmerkung: An der Tafel Stand nicht E -> E' sondern E->V'; im einer Zusammenfassung von Algebra im ET Forum stand diese Version. Was ist richtig?
\\D.h. das Bild des Anfangsknoten einer Kante e ist Anfangsknoten der Bildkante.
\paragraph{Merkregel}``Fuß und Kopf gehen nicht verloren''


%Die nächsten Zeilen müssen noch an passender Stelle im Script eingefügt werden
%\subsubsection{Beispielklasse: Binäre Relation als Netzwerke}

%\paragraph{Def:} Eine binäre Relation ist erklärt als Paar $M:=(M,R)$
%mit M und R sind Mengen und es ist $R\subseteqq MxM$, d.h R ist
%``binäre Relation auf M''.

%Sei $GM := (M,R,\sigma,\tau)$ mit $\sigma : E \rightarrow V, (p,q) \mapsto p$ und $\tau : E \rightarrow V, (p,q) \mapsto q$

%GM heiße den zu M gehörige Netzwerke bzw.  ``M=(M,R) als Netzwerke''