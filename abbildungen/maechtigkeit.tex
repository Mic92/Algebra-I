\subsection{Mächtigkeit von Mengen von Abbildungen}
Seien $A$ und $B$ Mengen.
Dann bezeichnet $B^A$ oder $\Map(A,B)$ die Menge aller Abbildungen von $A$
nach $B$. \paragraph*{Satz:}
Für $A, B$ endliche Mengen gilt:
$$ |B^A| = {|B|}^{|A|} $$

\subsubsection{Formeln zur Berechnung der Mächtigkeit von Mengen von Abbildungen}
\paragraph*{bijektive Abbildungen}
Sei $|A|=|B|$ endlich. Die Mächtigkeit der Menge aller bijektiven
Abbildungen von A nach B entspricht der Anzahl der Permutationen auf
einer |A|-elementigen Menge.
\begin{align*}
  A &\rightarrow B \\
  |Bij(A,B)| = (|A|)! &= (|B|)!
\end{align*}

\paragraph*{injektive Abbildungen}
Seien |A| und |B| endlich. Die Mächtigkeit der Menge aller injektiven
Abbildungen von A nach B entspricht einer \emph{Variation} ohne Zurücklegen:
\begin{align*}
  A &\rightarrow B \\
  a &= |A| \; b = |B| \\
  |Inj(A,B)| =
{{b}\choose{a}} * a!
  &= \frac{b!}{(b-a)!}
\end{align*}

\subparagraph*{Anmerkung:} Für den Sonderfall $|A|=|B|$ führt die Formel auf die Formel für die bijektiven Abbildungen zurück. Das bedeutet, dass im Fall $|A|=|B|$ endlich jede Injektion
automatisch auch eine Bijektion ist:
$$|A|=|B| \Longleftrightarrow Inj(A,B) = Bij(A,B) = Surj(A,B)$$
z.B.: $ Inj(A,A) = Bij(A,A) =: Perm(A) $
(die Permutation einer Menge ist die Bijektion einer Menge in sich selbst)

\paragraph*{surjektive Abbildung}
Seien |A| und |B| endlich. Die Mächtigkeit der Menge aller surjektiven
Abbildungen von A nach B lässt sich mithilfe der
Stirlingzahl 2. Art ($S_{n,r}$) berechnen.
\begin{align*}
  A &\rightarrow B \\
  a &= |A| \; b = |B| \\
  |Surj(A,B)| =
  b!    &\cdot S_{(a,b)}\\
  S(a,b)&=\frac{1}{b!}\sum_{j=1}^{b}(-1)^{b-j}{b \choose j}j^a
\end{align*}

\subparagraph*{Bemerkung:} Die Formel für surjektive Abbildung sollte als Ergänzung zum Vorlesungsstoff verstanden werden.
Sie wird(soweit wir bisher wissen) nicht als bekannt vorrausgesetzt.

\subparagraph{Weitere Bemerkungen}
Es gilt:
$ |A| = n \Longleftrightarrow Bij( \{1, ... , n\},A) \neq \emptyset$

\subsubsection{Herleitung einer rekursiven Formel für |Surj(A,B)| }
Idee: $$ B^A =  \underset {{X}\subseteq{B}} {\mathbin{\dot{\cup}}} {\{ f\in B^A | f(A) = X \}} $$

Offensichtlich gilt: $$ | {f \in B^A | f(A) = X} | = |Surj(A,X)| = s(|A|, |X|) $$
Da $ \{ \{f \in B^A | f(A) = \lambda \} X \subseteq B \} $ eine Partition(d.h. ``disjunkte Zerlegung``) von $B^A$ ist, gilt nun:

$$ {|B|}^{|A|} = |{B}^{A}| = \sum_{x \subseteq B} s(|A|,|X|)
= \sum_{i = 0}^{|B|} \sum_{X \in  {{|B|}\choose{i}} } s(|A|,i) =
\sum_{i = 0}^{|B|}   {{|B|}\choose{i}} *  s(|A|,i)
$$

Es folgt mit $n:=|A|$ sowie $k:=|B|$ :
$$ \sum_{i = 0}^{k}   {{k}\choose{i}} *  s(n,i)  =
{{k}\choose{0}} * s(n,0) + {{k}\choose{1}} * s(n,1) + {{k}\choose{2}} * s(n,2) + \dots + {{k}\choose{k}} * s(n,k) = k^n $$

Weiterhin gilt: $ s(0,0) = 1 $ und $s(n,1)=1$ sowie $ s(n,0)=0 $ für $ {n}\neq{0} $

\subsubsection{Anwendung der rekursiven Formel}
Exemplarisch soll am Beispiel k=3 die Berechnung mit einer rekursiven Formel gezeigt werden.
Die im vorherigen Abschnitt hergeleitete Formel gibt nur implizit s(n,k) an, als erster Schritt soll die Angabe explizit geschehen.
Das heißt, die Gleichung wird zunächst einfach umgestellt:
$$ s(n,k) = k^n - {{k}\choose{1}} * s(n,1) - {{k}\choose{2}} * s(n,2) + ... + {{k}\choose{k-1}} * s(n,k-1) $$
Also gilt für k = 3:
$$ s(n,3) = 3^n - {{3}\choose{1}} * s(n,1) - {{3}\choose{2}} * s(n,2) $$

Offensichtlich müssen zur Berechung eines Elementes erst die vorherigen Elemente bestimmt werden:
$$ s(n,1) = 1$$
$$ s(n,2) = 2^n - {{2}\choose{1}} * s(n,1) = 2^n -2 $$

Die Ergebnisse können nun eingesetzt werden:
$$ s(n,3) = 3^n - {{3}\choose{1}} * 1 - {{3}\choose{2}} * (2^n -2) = 3^n - 3 * 2^n + 3 $$

So ergibt sich z.B. für n = 5:
$$ s(5,3) = 3^5 - 3 * 2^5 + 3 = 150 $$